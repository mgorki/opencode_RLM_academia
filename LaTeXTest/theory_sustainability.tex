\documentclass{article}
\usepackage[utf8]{inputenc}
\usepackage[style=apa,backend=biber]{biblatex}
\addbibresource{refs.bib}
\begin{document}

\section{Theory}
\subsection{Relationship between Sustainability and Sustainable Development}

The literature distinguishes two principal ways in which the concepts of *sustainability* and *sustainable development* are treated. Some scholars use the terms interchangeably, whereas a substantial body of work stresses subtle but important differences.

\begin{itemize}
  \item \textbf{Interchangeability.} Stefańska (2021) notes that in many Polish publications the two terms are translated identically, leading authors to treat them as synonymous \parencite{stefaska2021sustainability}. This “interchangeable” approach dominates much of the existing scholarship \parencite{stefaska2021sustainability}.
  \item \textbf{Distinction.} A second, more nuanced view separates the concepts on the basis of the “growth” connotation of development. Sustainable development is framed as a set of development goals that pursue long‑term economic success while safeguarding the capacity of future generations \parencite{stefaska2021sustainability}. By contrast, *sustainability* is described as a property of systems that enables them to endure over time. Stefańska (2021) defines sustainability as “the relationship between dynamic human economic systems and larger dynamic, but normally slower, changing ecological systems, in which (a) human life can continue indefinitely, (b) human individuals can flourish, and (c) human cultures can develop” \parencite{stefaska2021sustainability}.
  \item \textbf{Processual link.} The same source argues that sustainability “is a concept directly leading to another one—sustainable development, which, in turn, sheds more light on the needs of future generations” \parencite{stefaska2021sustainability}. Thus, sustainability is often portrayed as the normative principle that constrains and informs the trajectory of development.
\end{itemize}

Connelly (2007) treats sustainable development as an essentially contested concept situated at the intersection of environmental, economic, and social dimensions of sustainability, emphasizing that the term lacks a fixed boundary and is shaped by political debate \parencite{connelly2007mapping}.

The distinction is echoed in broader reviews of the field. Ruggerio (2021) reports that the WCED (1987) defined sustainable development as “development that meets the needs of the present without compromising the ability of future generations to meet their own needs” \parencite{ruggerio2021}; the author further observes a persistent ambiguity between *sustainable growth* and *sustainable development* and stresses the need to separate the two concepts when discussing policy and governance \parencite{ruggerio2021}.

A recent systematic exploration likewise draws a clear line between the two notions. Bhurat \& Thakrar (2023) describe sustainability as “how to manage resources so that future generations can continue to use them” and argue that it “includes social and economic sustainability, both of which entail addressing the present‑day social and economic wants of people while not ignoring the needs and well‑being of subsequent generations” \parencite{bhurat2023exploration}. In contrast, they define sustainable development as “the aim to raise long‑term economic success and standard of living without endangering the capacity of posterity to meet their own requirements” \parencite{bhurat2023exploration}.

\paragraph{Synthesis}
Taken together, the corpus suggests a consensus that:
\begin{enumerate}
  \item *Sustainability* denotes a condition or capacity of a system (social, economic, ecological) to persist over time, emphasizing balance, resilience, and intergenerational equity.
  \item *Sustainable development* denotes a developmental trajectory or set of policies that seek to improve human well‑being while respecting the limits identified by sustainability.
  \item The two concepts are hierarchically linked: sustainability provides the normative boundary within which sustainable development must operate.
\end{enumerate}

These perspectives underline why many scholars favour treating the terms as distinct yet mutually reinforcing, especially in policy formulation and interdisciplinary research.

\printbibliography
\end{document}